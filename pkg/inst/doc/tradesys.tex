\documentclass[a4]{article}
\usepackage{G:/robert/tex/sty/Sweave} % Mayfair machine needs this..
%\usepackage{Sweave} 
\usepackage{url}

\newcommand{\code}[1]{\texttt{#1}}

\begin{document}
\author{Robert Sams}
\title{\code{tradesys}: A framework for encoding and backtesting trading systems in R}
\maketitle


%% \VignetteIndexEntry{tradesys: A framework for encoding and backtesting trading systems in R}
%% \VignetteDepends{zoo}
%% \VignetteKeywords{trading systems, backtesting}
%% \VignettePackage{tradesys}

%%%%%%%%%%%%%%%%%%%%%%%%%%%%%%%%%%%%%%%%%%%%%%%%%%%%%%%%%%%%%%%%%%%%%%%
\section{Introduction} \label{sec:intro}
%%%%%%%%%%%%%%%%%%%%%%%%%%%%%%%%%%%%%%%%%%%%%%%%%%%%%%%%%%%%%%%%%%%%%%%
The \code{tradesys} package is for modelling trading systems in R. The
key functionality of the package is centered around the 'tsys'
class. A 'tsys' object collects all of the information needed to
completely define a trading system. The key ideas behind the system,
like the entry and exit signals, are written by the user in R and
stored as unevaluated R expressions in the appropriate 'tsys'
slot. The object can subsequently be applied to any timeseries data
with the appropriate variables, the logic of the system encoded in the
stored expressions are evaluated on this data, and the system's
long/short/flat \emph{states} and the resulting \emph{equity curve}
are calculated. In short, a trading system consists of certain
R-encoded logic and meta-data, defined in an object of class 'tsys',
and evaluated on any data of the appropriate structure to calculate
the system's states and equity.

As the above description implies, \code{tradesys} is about
\emph{modelling} a trading system, not analyzing its \emph{results},
the latter being what is properly referred to as ``backtesting''....

So, the functionality of the package is modest in its scope. It is,
however, ambitious in the implementation. The above model is needed in
almost all trading system research and thus represents a problem in
need of a common, well-designed solution. This package aims to do this
to the highest standard, so that trading system builders who chose R
as an important tool of analysis can confidently use this package a
key component of their work.

There are three main design goals of the package. First, the package
should be \emph{trustworthy}, in the sense elaborated by Chambers 

Second, \emph{discoverability}. The logic behind every calculation
should not only be scrupulously documented, but also discoverable, in
the sense that the key computations done on 'tsys' objects are
encapsulated in functions that the user can call and explore. So the
package contains a number of functions that are not strictly speaking
part of the user interface (the usage of the package can be had
without ever calling them), but are documented seperately and exported
by the package's namespace to be explored at will.

%%%%%%%%%%%%%%%%%%%%%%%%%%%%%%%%%%%%%%%%%%%%%%%%%%%%%%%%%%%%%%%%%%%%%%%
\section{A formal definition of ``trading system''}
%%%%%%%%%%%%%%%%%%%%%%%%%%%%%%%%%%%%%%%%%%%%%%%%%%%%%%%%%%%%%%%%%%%%%%%
A \emph{trading system} is an algorithm on a timeseries $X_{t}$ that
specifies, for each time \emph{t}, whether the system's state is long,
short or flat. Mathematically, it is a function $f(X_{t})$ that
calculates each state $s_{i} \in \left\{1,0,-1\right\}$ on the basis
of $X_{1}, ..., X_{i}$. $X_{t}$ may be as simple as a daily series of
closing prices but is often a multivariate series with various price
and other data. The states vector combined with the timeseries is the
raw material for backtesting research from the calculation of period
returns onwards. Let's call such a combination $\left\{X_{t},
s_{t}\right\}$ a \emph{trading system time series}. In this package a
trading system time series is represented as class \code{tsts}.

But what about $f(X_{t})$, what form does it take?...

%%%%%%%%%%%%%%%%%%%%%%%%%%%%%%%%%%%%%%%%%%%%%%%%%%%%%%%%%%%%%%%%%%%%%%%
\section{Introductory Examples}
%%%%%%%%%%%%%%%%%%%%%%%%%%%%%%%%%%%%%%%%%%%%%%%%%%%%%%%%%%%%%%%%%%%%%%%

%%%%%%%%%%%%%%%%%%%%%%%%%%%%%%%%%%%%%%%%%%%%%%%%%%%%%%%%%%%%%%%%%%%%%%%
\section{Splicing Timeseries}
%%%%%%%%%%%%%%%%%%%%%%%%%%%%%%%%%%%%%%%%%%%%%%%%%%%%%%%%%%%%%%%%%%%%%%%

%%%%%%%%%%%%%%%%%%%%%%%%%%%%%%%%%%%%%%%%%%%%%%%%%%%%%%%%%%%%%%%%%%%%%%%
\section*{Computational details}
%%%%%%%%%%%%%%%%%%%%%%%%%%%%%%%%%%%%%%%%%%%%%%%%%%%%%%%%%%%%%%%%%%%%%%%
The results in this paper were obtained using R
2.8.0 with the packages
\code{tradesys} 0.1 
and \code{zoo}  1.5--4 R itself
and all packages used are available from CRAN at
\url{http://CRAN.R-project.org/}.

\end{document}


