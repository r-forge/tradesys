\documentclass[a4]{article}
%\usepackage{G:/robert/tex/sty/Sweave} % Mayfair machine needs this..
\usepackage{Sweave} 
\usepackage{url}

\newcommand{\code}[1]{\texttt{#1}}

\begin{document}
\author{Robert Sams}
\title{\code{tradesys}: A framework for modelling trading systems in R}
\maketitle


%% \VignetteIndexEntry{tradesys: A framework for encoding and backtesting trading systems in R}
%% \VignetteDepends{zoo}
%% \VignetteKeywords{trading systems, backtesting}
%% \VignettePackage{tradesys}

%%%%%%%%%%%%%%%%%%%%%%%%%%%%%%%%%%%%%%%%%%%%%%%%%%%%%%%%%%%%%%%%%%%%%%%
\section{Introduction} \label{sec:intro}
%%%%%%%%%%%%%%%%%%%%%%%%%%%%%%%%%%%%%%%%%%%%%%%%%%%%%%%%%%%%%%%%%%%%%%%
\begin{quote}
Define your project and the right tool appears. Questions about the
tools indicate uncertainty about the project. \\Ed Seykoda
\end{quote}

The \code{tradesys} package is for modelling trading systems in R. The
key functionality of the package is centered around the 'tsys'
class. A 'tsys' object collects all of the information needed to
completely define a trading system. The key ideas behind the system,
like the entry and exit signals, are written by the user in R and
stored as unevaluated R expressions in the appropriate 'tsys'
slot. The object can subsequently be applied to any timeseries data
with the appropriate variables, the logic of the system encoded in the
stored expressions are evaluated on this data, and the system's
long/short/flat \emph{states} and the resulting \emph{equity curve}
are calculated. In short, a trading system consists of certain
R-encoded logic and meta-data, defined in an object of class 'tsys',
and evaluated on any data of the appropriate structure to calculate
the system's states and equity.

So, the functionality of the package is modest in its scope. It is,
however, ambitious in the implementation. The above model is needed in
almost all trading system research and thus represents a problem in
need of a common, well-designed solution. This package aims to do this
to the highest standard, so that trading system builders who chose R
as an important tool of analysis can confidently use this package a
key component of their work.

There are three main design goals of the package, one bold, two
conservative. First, the bold. The package aims at \emph{maximum
expressibility}. There is no one model that captures everything that
is properly thought of as a trading system, but many (most?) trading
systems share a common model and the ``tsys'' class aims to represent
this in R. This goal is cast in the concept of ``expressible'' rather
that ``feature-full'' because the class is designed with the notion
that seemingly special aspects of the system that you want to model
can be expressed in terms of simplier, basic ideas. For example, there
is no explicit modelling of slippage assumptions in ``tsys'', yet the
mechanism for putting data variables into a pricing context implicitly
allows the user to express just about any slippage model he
wants. We'll show some slightly more exotic examples later in this
document demonstrating how ideas like slippage, financing costs,
dynamic hedging, etc can be modelled in the class by building on its
simple but flexible components. (Users are encouraged to construct
examples that challenge this framework so that this approach can be
extended an improved in later versions.)

Second, the package should be \emph{trustworthy}, in the sense
elaborated by Chambers .... This is a conservative goal but arguably
more important than the one just mentioned...

Third, \emph{discoverability}. The logic behind every calculation
should not only be scrupulously documented, but also discoverable, in
the sense that the key computations done on 'tsys' objects are
encapsulated in functions that the user can call and explore. So the
package contains a number of functions that are not strictly speaking
part of the user interface (the usage of the package can be had
without ever calling them), but are documented seperately and exported
by the package's namespace to be explored at will.

\subsection{What about ``backtesting''?}
The \code{tradesys} package is \emph{not} a ``framework for
backtesting''. One often hears requests for such a thing among R users
in finance. But there is nothing meaningful about a test on back data
without a conjecture about some market anomoly and some ideas about a
system that can exploit it with an edge. The appropriate tests are as
varied as the conjectures, and so with the tools needed for the
testing. 

As it so happens, the tools needed backtesting properly conceived are
either computationally easy well-supported in R (see the various
packages for timeseries analysis, robust methods and, of course, the
base language); it's formulating the conjectures and key tests that
are hard. Hence, the Ed Seykoda comment opening this document.

Modelling the trading system, on the other hand, is conceptually easy
but computationally tricky. Coding from scratch the relationship
among a system's signals, states, equity curve, etc. is not only
time-consuming but pregnant with risks of subtle, logical
errors. Making this part easy and trustworthy is the motivation for
\code{tradesys}. 

But if what you are after is a mechanism for translating a system into
your favourite suite of ``performance measures'', this package does
99\% of what you need: just call your favourite functions on a 'tsys'
object's equity curve. But please don't look here for the home of R's
latest implementation of the Sharpe Ratio.

%%%%%%%%%%%%%%%%%%%%%%%%%%%%%%%%%%%%%%%%%%%%%%%%%%%%%%%%%%%%%%%%%%%%%%%
\section{A formal definition of ``trading system''}
%%%%%%%%%%%%%%%%%%%%%%%%%%%%%%%%%%%%%%%%%%%%%%%%%%%%%%%%%%%%%%%%%%%%%%%
A \emph{trading system} is an algorithm on a timeseries $X_{t}$ that
specifies, for each time \emph{t}, whether the system's state is long,
short or flat. Mathematically, it is a function $f(X_{t})$ that
calculates each state $s_{i} \in \left\{1,0,-1\right\}$ on the basis
of $X_{1}, ..., X_{i}$. $X_{t}$ may be as simple as a daily series of
closing prices but is often a multivariate series with various price
and other data. The states vector combined with the timeseries is the
raw material for backtesting research from the calculation of period
returns onwards. Let's call such a combination $\left\{X_{t},
s_{t}\right\}$ a \emph{trading system time series}. In this package a
trading system time series is represented as class \code{tsts}.

But what about $f(X_{t})$, what form does it take?...

%%%%%%%%%%%%%%%%%%%%%%%%%%%%%%%%%%%%%%%%%%%%%%%%%%%%%%%%%%%%%%%%%%%%%%%
\section{Introductory Examples}
%%%%%%%%%%%%%%%%%%%%%%%%%%%%%%%%%%%%%%%%%%%%%%%%%%%%%%%%%%%%%%%%%%%%%%%

%%%%%%%%%%%%%%%%%%%%%%%%%%%%%%%%%%%%%%%%%%%%%%%%%%%%%%%%%%%%%%%%%%%%%%%
\section{Splicing Timeseries}
%%%%%%%%%%%%%%%%%%%%%%%%%%%%%%%%%%%%%%%%%%%%%%%%%%%%%%%%%%%%%%%%%%%%%%%

%%%%%%%%%%%%%%%%%%%%%%%%%%%%%%%%%%%%%%%%%%%%%%%%%%%%%%%%%%%%%%%%%%%%%%%
\section*{Computational details}
%%%%%%%%%%%%%%%%%%%%%%%%%%%%%%%%%%%%%%%%%%%%%%%%%%%%%%%%%%%%%%%%%%%%%%%
The results in this paper were obtained using R
2.9.1 with the packages
\code{tradesys} 0.1 
and \code{zoo}  1.5--6 R itself
and all packages used are available from CRAN at
\url{http://CRAN.R-project.org/}.

\end{document}


